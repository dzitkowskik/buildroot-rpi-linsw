\documentclass{mini}
\usepackage[utf8]{inputenc}
\usepackage[polish]{babel}
\usepackage{enumitem}
\usepackage{todonotes}
\presetkeys
	{todonotes}
	{inline}{}

%------------------------------------------------------------------------------%
\title{Linux w Systemach Wbudowanych}
\author{Karol Dzitkowski}
\monthyear{\today}
%------------------------------------------------------------------------------%

\usepackage{listings}
\begin{document}
\maketitle
\tableofcontents
\newpage

\section{Treść zadania}
\begin{enumerate}
\item Przygotować obraz systemu Linux, używający ,,initramfs'' jako głównego systemu plików.
\item RPi powinno automatycznie podłączyć się do sieci, używając DHCP do otrzymania parametrów sieci
\item System powinien mieć ustaloną nazwę jako: ,,imię nazwisko'' autora
\item Obraz systemu powinien zawierać dodatkowe programy wskazane przez prowadzącego (powiedzmy, że na przykład chcemy, żeby był tam prosty serwer WWW udostępniający statyczne strony HTML)
\item System powinien synchronizować czas systemowy z serwerem czasu
\item Hasło użytkownika ,,root'' oraz hasło użytkownika domyślnego ,,default'' powinno być odpowiednio ustawione przy starcie systemu.
\end{enumerate}

\section{Rozwiązanie}

\subsection{Konfiguracja \emph{buildroot}}
\begin{enumerate}
	\item Klonuję repozytorium do \emph{/malina/buildroot}
	\begin{lstlisting}[language=bash]
	git clone git://git.buildroot.net/buildroot --depth 1
	\end{lstlisting}
	\item Przechodzę do utworzonego katalogu
	\begin{lstlisting}[language=bash]
	cd buildroot
	\end{lstlisting}	
	\item Ustawiam konfigurację dla Raspberry Pi
	\begin{lstlisting}[language=bash]
	make raspberrypi_defconfig
	\end{lstlisting}	
	\item Wpisuję \emph{make menuconfig} aby przejść do konfiguracji i ustawiam:
	\begin{itemize}	
		\item Filesystem images (pkt 1 zadania)
		\begin{itemize}
			\item initial RAM filesystem linked into linux kernel
		\end{itemize}
		
		\item System configuration
		\begin{itemize}
			\item Root filesystem overlay directories ustawiam na \emph{moje}
			\item System hostname ustawiam na \emph{Karol Dzitkowski}
			\item Root password ustawiam na \emph{password}
			\item Network interface to configure through DHCP
			\item Install timezone info
				\begin{itemize}
				\item Timezone list ustawiam na \emph{default}
				\item Default local time ustawiam na \emph{Europe/Warsaw}
				\end{itemize}
			\item Path to the users tables ustawiam na \emph{board/linsw}
			\item Getty options 
			\begin{itemize}
				\item TTY port ustawiam na (\emph{ttyAMA0})
			\end{itemize}
		\end{itemize}
		
		\item Packages 
		\begin{itemize}
			\item ntp i ntpdate oraz dodatkowo ntpdc
			\item dhcp client (domyślnie w \emph{busybox})
			\item dropbear do ustawienia ssh
			\item lighttpd
		\end{itemize}
	\end{itemize}
	
	\item Tworzę i wypełniam plik definiujący konto dodatkowego użytkownika
	\begin{lstlisting}[language=bash]
	echo "default -1 default -1 =password /home/default /bin/sh -" \
	> /board/linsw
	\end{lstlisting}	
		
	\item Tworzę katalogi do overlay
	\begin{lstlisting}[language=bash]
	mkdir moje
	mkdir moje/etc
	mkdir moje/var/www
	mkdir moje/etc/init.d
	mkdir moje/etc/network
	\end{lstlisting}	
	
	\item Tworzę i wypełniam plik strony www
	\begin{lstlisting}[language=bash]
echo "<html><p>Hello World!</p></html>" > moje/var/www/index.html 
	\end{lstlisting}
		
	\item Aktualizuję konfigurację sieci w pliku interfaces
	\begin{lstlisting}[language=bash]
	vim moje/etc/network/interfaces
	...
	\end{lstlisting}
		
	\item Tworzę plik uruchamiany zaraz po starcie systemu w celu skonfigurowania dhcp oraz zamontowania karty SD (do kopiowania kernela przez ssh)
	\begin{lstlisting}[language=bash]
	vim moje/etc/init.d/S99start
	#! /bin/sh
	sleep 5
	udhcpc
	mkdir -p /media/card
	mount /dev/mmcblk0p1 /media/card
	\end{lstlisting}
		
	\item Buduję system
	\begin{lstlisting}[language=bash]
	make
	\end{lstlisting}		
\end{enumerate}

\subsection{Uruchamianie \emph{Raspberry}}

\begin{enumerate}
\item Kopiuję pliki z katalogu output/images na wcześniej sformatowaną kartę pamięci (MS-DOS FAT na OS X):	\begin{lstlisting}[language=bash]
cp output/images/rpi-firmware/* <katalog karty SD>
cp output/images/zImage <katalog karty SD>
\end{lstlisting}	

\item Edytuję plik ... i ustawiam tty na ttyAMA0
\item Podłączam Raspberry do sieci poprzez użycie Ethernet, do komputera przez port szeregowy z przejściówką na USB oraz do prądu używając zasilania z USB komputera

\end{enumerate}
\section{Praca z urządzeniem}
\begin{enumerate}
\item Używając narzędzia \emph{minicom} podłączam się do konsoli urządzenia
\begin{lstlisting}[language=bash]
minicom -D \dev\ttyUSB0 -o
\end{lstlisting}	
Zaraz po tym wciskam Ctrl-A w celu wyłączenia sprzętowej kontroli przepływu (hardware flow control) w ustawieniach portu szeregowego
\item Raspberry pi startuje, montuje kartę SD, włącza sieć używając udhcpc
\item Wpisuję nazwę użytkownika root lub default oraz hasło password
\item W wypadku przerwania połączenia z internetem muszę ponownie wpisać udhcpc w konsoli
\item Wpisanie polecenia date skutkuje wypisaniem poprawnego lokalnego czasu
\item Wpisuję ifconfig i zapamiętuję nadany adres IP na eth0
\item Sprawdzam na komputerze stronę internetową na zapamiętanym adresie IP
\item Ustawiam połączenie SSH na komputerze stacjonarnym
\begin{lstlisting}[language=bash]
cd ~/ssh
scp root@<zapamietany adres IP>:/etc/dropbear/dropbear_ecdsa_host_key ./
yes, haslo
ssh root@<zapamietany adres IP>
\end{lstlisting}
Teraz mogę już śmiało korzystać z Raspberry i przy zmianach kernela podmieniać go przez ssh (scp)
\end{enumerate}

\section{Wygenerowany plik konfiguracyjny buildroot}
\begin{lstlisting}[language=bash]
cat configs/raspberrypi_defconfig 

\end{lstlisting}	
\end{document}
