\documentclass{mini}
\usepackage[utf8]{inputenc}
\usepackage[polish]{babel}
\usepackage{enumitem}
\usepackage{todonotes}
\presetkeys
	{todonotes}
	{inline}{}

%------------------------------------------------------------------------------%
\title{Linux w Systemach Wbudowanych}
\author{Karol Dzitkowski}
\monthyear{\today}
%------------------------------------------------------------------------------%

\usepackage{listings}
\begin{document}
\maketitle
\tableofcontents
\newpage

\section{Treść zadania}
\begin{enumerate}
\item Przygotować ,,administracyjny'' system Linux pracujący w
initramfs, umożliwiający przygotowanie karty pamięci SD do
instalacji systemu Linux pracującego z systemem plików
$e2fs$, montowanym z partycji 2 na karcie SD

\item Przygotować „użytkowy” system Linux pracujący z
systemem plików $e2fs$, zawierający serwer WWW,
udostępniający pliki z partycji 3 na karcie SD i umożliwiający
wgrywanie nowych plików po podaniu hasła.

\item Należy też przygotować bootloader, umożliwiający
określenie (przy pomocy przycisku), który system ma zostać
załadowany.
\end{enumerate}

\section{Rozwiązanie}
Stworzyłem dwie wersje systemu linux używając buildroota. jeden administracyjny 
z systemem plików initramfs oraz narzędziami to partycjonowania i tworzenia systemu plików $e2fs$. 
Drugi dla użytkownika pracujący z systemem plików $ext4$ i z wbudowanym pythonem oraz obsługą GPIO.
Dodatkowo napisałem serwer webowy uzywając biblioteki pythona Tornado,
udostępniający pliki z folderu tymczasowego i umożliwiający upload 
plików po podaniu hasła. Partycje zostały stworzone za pomocą systemu administracyjnego 
używając gptfdisk i $mkfs.ext4$. Ponadto użyłem bootloadera
barebox który umożliwia wybranie systemu lub wejście do bootloadera.
Jeżeli system nie zostanie wybrany w ciągu 10 sekund, zostanie załadowany system użytkownika.

\subsection{Konfiguracja \emph{buildroot}}
Bazuję na konfiguracji z zadania pierwszego i dodatkowo w menu wywołanym za pomocą komendy:

\begin{lstlisting}[language=bash]
make menuconfig
\end{lstlisting}	

Dla systemu użytkownika ustawiam:
\begin{enumerate}
\item Filesystem images $\rightarrow$ ustawiam $ext2/3/4$ z wariantem $ext4$ i zmieniam filesystem label na \emph{rootfs}
\item Kernel $\rightarrow$ ustawiam Install kernel image to /boot in target
\item Bootloaders $\rightarrow$ ustawiam Barebox i zmieniam board defconfig na \emph{rpi}
\item Target packages $\rightarrow$ Interpreter languages and scripting $\rightarrow$ ustawiam python oraz w external python modules ustawiam \emph{python-tornado}
\end{enumerate}

Dla systemu administracyjnego ustawiam:
\begin{enumerate}
\item Filesystem images $\rightarrow$ ustawiam initial RAM filesystem oraz Kernel binary format na uImage
\item Target packages $\rightarrow$ Filesystem and flash utilities $\rightarrow$ ustawiam e2fsprogs
\item Target packages $\rightarrow$ Hardware handling $\rightarrow$ ustawiam gptfdisk 
\end{enumerate}

\section{Partycjonowanie}


\section{Konfiguracja barebox}
\begin{itemize}
\item Zmieniam w pliku config.txt na karcie w partycji sformatowanej jako vfat, ustawiając
$$kernel=barebox.bin$$
\item Po poprawnym sformatowaniu karty kopiuję zawartość pliku init (z repozytorium w folderze scripts zadania 3) do pliku $ /env/bin/init $ zastępując poprzednią zawartość. Po czym wykonuję komendy:
\begin{lstlisting}[language=bash]
$ saveenv
$ reset
\end{lstlisting}	
\end{itemize}

\section{Pliki konfiguracyjne i źródła}
Pliki konfiguracyjne i źródła do tego jak i wcześniejszych zadań umieściłem w repozytorium
git na serwisie www.github.com. Aby ściągnąć te źródła należy w katalogu docelowym wykonać polecenie:
\begin{lstlisting}[language=bash]
git clone https://github.com/dzitkowskik/buildroot-rpi-linsw.git
\end{lstlisting}	
\begin{itemize}
\item buildroot\_adm\_conf -- pliki konfiguracyjne do systemu administracyjnego na initramfs
\item buildroot\_user\_conf -- pliki systemu użytkownika z systemem plików $ext4$
\item file\_server -- server do udostępniania plików napisany w pythonie (user: admin, hasło: admin)
\item scripts -- skrypty (init - skrypt startowy dla barebox, update.sh - skrypt shellowy do aktualizacji systemu plików $e2fs$ na partycji dla użytkownika)
\item docs -- dokumentacja
\end{itemize}
\end{document}
