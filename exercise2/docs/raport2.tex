\documentclass{mini}
\usepackage[utf8]{inputenc}
\usepackage[polish]{babel}
\usepackage{enumitem}
\usepackage{todonotes}
\presetkeys
	{todonotes}
	{inline}{}

%------------------------------------------------------------------------------%
\title{Linux w Systemach Wbudowanych}
\author{Karol Dzitkowski}
\monthyear{\today}
%------------------------------------------------------------------------------%

\usepackage{listings}
\begin{document}
\maketitle
\tableofcontents
\newpage

\section{Treść zadania}
\begin{enumerate}
\item Przygotować aplikację w języku C, obsługującą przyciski i diody LED.
\item Aplikacja powinna reagować na zmiany stanu przycisków bez oczekiwania aktywnego
\item Funkcjonalność aplikacji, może być ustalona przez studenta przykłady 
(można wymyślić własną, ciekawszą aplikację): stoper, z LED-ami potwierdzającymi operację i
z możliwością pomiaru „międzyczasów” bądź zamek szyfrowy z logowaniem prób otwarcia. LED-y sygnalizują:
otwarcie zamka, błędny kod, alarm po kilkukrotnym wprowadzeniu błędnego kodu.
\item Przetestować korzystanie z debuggera (gdb) przy uruchamianiu aplikacji
\item Zagadnienie dodatkowe: przygotować równoważną aplikację w wybranym języku skryptowym i porównać
wydajność obu implementacji
\end{enumerate}

\section{Rozwiązanie}
Napisałem aplikację w języku C++ a następnie taką samą w Pythonie, która symulowała zamek na kod pin.
Kod wstukuje się za pomocą przycisków dopóki nie zapalą się wszystkie diody LED. Następnie kod 
zapisywany jest w pliku. Po ponownym uruchomieniu programu, program pyta się o zapisane hasło. Jeśli
wciśniemy tą samą konfigurację przycisków diody się zaświecą i wypisze się komunikat o sukcesie oraz
plik z hasłem zostanie usunięty. W przeciwnym wypadku zostaniemy poinformowani o porażce.

\subsection{Konfiguracja \emph{buildroot}}
Bazuję na konfiguracji z zadania poprzedniego i dodatkowo w menu wywołanym za pomocą komendy:
	\begin{lstlisting}[language=bash]
make menuconfig
	\end{lstlisting}	
ustawiam:
\begin{enumerate}
	\item W Target Packages -> Interpreter languages and scripting, wybieram python
	\item W Target Packages -> Debugging ..., wybieram gdb (gdbserver)
	\item W Toolchain -> Toolchain type, wybieram External toolchain
	\item W Toolchain ustawiam również Build cross gdb for the host
	\item W Build options ustawiam Build packages with debugging symbols
\end{enumerate}

Ponadto w konfiguracji jądra do którego wchodzę za pomocą komendy:
	\begin{lstlisting}[language=bash]
make linux-menuconfig
	\end{lstlisting}	
ustawiam:
\begin{enumerate}
	\item Device Drivers -> GPIO Support -> /sys/class/gpio/... (sysfs interface)
\end{enumerate}

następnie na początku pliku /malina/buildroot/package/Config.in dodaję:
\begin{lstlisting}[language=bash]
menu "My menu"
source "package/ledkeys/Config.in"
endmenu
\end{lstlisting}

co więcej dodałem katalog ledkeys do katalogu package z dwoma plikami:
\begin{itemize}
	\item Config.in
	\begin{lstlisting}[language=bash]
config BR2_PACKAGE_LEDKEYS
bool "ledkeys"
help
	Keys game with leds
\end{lstlisting}
	\item keys.mk
\begin{lstlisting}[language=bash]
LEDKEYS_VERSION = 1.0
LEDKEYS_SITE = $(TOPDIR)/myPackages/keys
LEDKEYS_SITE_METHOD = local

define LEDKEYS_BUILD_CMDS
	$(MAKE) $(TARGET_CONFIGURE_OPTS) keys -C $(@D)
endef
define LEDKEYS_INSTALL_TARGET_CMDS
	$(INSTALL) -D -m 0755 $(@D)/keys $(TARGET_DIR)/usr/bin
endef
LEDKEYS_LICENSE = GPL

$(eval $(generic-package))
\end{lstlisting}
\end{itemize}
Źródła mojego programu w C++ trzymam w katalogu /malina/buildroot/myPackages/keys wraz
z Makefile. Kod znajduje się w załączniku spakowanym wraz ze skryptem Pythona.
Na końcu:
\begin{enumerate}	
	\item Buduję system
	\begin{lstlisting}[language=bash]
	make
	\end{lstlisting}		
\end{enumerate}

\subsection{Debugowanie C++ na \emph{Raspberry}}
\begin{enumerate}
\item Poprzez terminal minicom na urządzeniu wpisałem polecenie:
\begin{lstlisting}[language=bash]
gdbserver host:7654 nano
\end{lstlisting}

\item Natomiast na stacji roboczej:
\begin{lstlisting}[language=bash]
/malina/buildroot/output/host/usr/bin/arm-none-linux-gnueabi-gdb /malina/buildroot/output/build/ledkeys-1.0/keys
\end{lstlisting}

\item Potem w sesji debuggera:
\begin{lstlisting}[language=bash]
(gdb) set sysroot /malina/buildroot/output/target
(gdb) target remote xxx.yyy.zzz.vvv:7654
\end{lstlisting}
gdzie $xxx.yyy.zzz.vvv$ oznacza adres IP mojego urządzenia.
\end{enumerate}

\section{Praca z urządzeniem}
\begin{enumerate}
\item Jako że użądzenie ma ustawiony serwer ssh łączę się do niego przy użyciu komendy:
\begin{lstlisting}[language=bash]
ssh root@<zapamietany adres IP>
\end{lstlisting}
aby nie mieć problemów z walidacją ssh wykonuję poniższe komendy:
\begin{lstlisting}[language=bash]
cd ~/ssh
scp root@<zapamietany adres IP>:/etc/dropbear/dropbear_ecdsa_host_key ./
yes, haslo
\end{lstlisting}
Teraz mogę już śmiało korzystać z Raspberry i przy zmianach kernela podmieniać go przez ssh (scp)
Mogę także przetestować działanie moich programów. Program napisany w C++ i zainstalowany przy pomocy
paczki buildroot uruchamiam wpisując:
\begin{lstlisting}[language=bash]
./keys
\end{lstlisting}
Aby uruchomić skrypt Pythona muszę go najpierw skopiować na kartę w raspberry:
\begin{lstlisting}[language=bash]
scp pythonKeys.py root@<zapamietany adres IP>:/mnt/sdcard/
chmod +x pythonKeys.py
./pythonKeys.py
\end{lstlisting}
Po testach doszedłem do wniosku że wszystko działa poprawnie.
\end{enumerate}

\section{Pliki konfiguracyjne i źródła}
Pliki konfiguracyjne i źródła do tego jak i wcześniejszego zadania laboratoryjnego umieściłem w repozytorium
git na serwisie www.github.com. Aby ściągnąć te źródła należy w katalogu docelowym wykonać polecenie:
\begin{lstlisting}[language=bash]
git clone https://github.com/dzitkowskik/buildroot-rpi-linsw.git
\end{lstlisting}	
\end{document}
